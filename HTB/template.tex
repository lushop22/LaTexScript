\documentclass[a4paper]{article} %Formato de plantilla 



\usepackage[utf8]{inputenc}
\usepackage[spanish]{babel}
\usepackage[margin=2cm, top=2cm, includefoot]{geometry}
\usepackage{graphicx} %Para la insercion de imagen
\usepackage{soul} %Para el sombreado de textos
\usepackage[table,xcdraw]{xcolor} %Para la deteccion de colores
\usepackage[most]{tcolorbox} %Para insercion se cuadros en la portada
\usepackage{fancyhdr} %Definir el estilo de la pagina
\usepackage[hidelinks]{hyperref} %Gestion de hipervinculos
\usepackage{listing} %para la insercion del codigo en el documento
\usepackage{parskip} %arreglo del a tabulacion en el documento 
\usepackage{smartdiagram} %Pala la insercion de diagramas
\usepackage{zed-csp} %Para la insercion de esquemas

%Declaracion de colores
\definecolor{greenPortada}{HTML}{599c3a}

%Declaracion de variables

\newcommand{\logoPortada}{DIRECCION_ACTUAL/HTBlogo.png}
\newcommand{\logoMachine}{DIRECCION_ACTUAL/MachineLogoExample.png}
\newcommand{\machineName}{Nombre de la Maquina} %Nombre de la maquina
\newcommand{\startDate}{Dia de hoy}


%Adicionales
\addto\captionsspanish{\renewcommand{\contentsname}{Indice}} %Cambio delformato del infice
\setlength{\headheight}{40.2pt}
\pagestyle{fancy}
\fancyhf{}
\lhead{\includegraphics[width=2.5cm]{\logoPortada}}\rhead{\includegraphics[height=1.6cm]{\logoMachine}}
\renewcommand{\headrulewidth}{3pt}
\renewcommand{\headrule}{\hbox to\headwidth{\color{greenPortada}\leaders\hrule height \headrulewidth\hfill}}


%Comienzo del documento
\begin{document}
  \cfoot{\thepage}
  % Creacion de Portada
  \begin{titlepage}
  \centering
    \includegraphics[width=0.4\textwidth]{\logoPortada}\par\vspace{1cm}
    {\scshape\LARGE \textbf{Apuntes importantes}\par}
    \vspace{0.2cm}
    {\Huge\bfseries\textcolor{greenPortada}{Maquina \machineName}\par}
    \vfill\vfill
    \includegraphics[width=\textwidth,height=9cm,keepaspectratio]{\logoMachine}\par\vspace{1cm}
    \vfill 
    \begin{tcolorbox}[colback=green!5!white,colframe=green!75!black]
      \centering
      Cuadro para colocar alguna informacion importante de la maquina o del
      informe
    \end{tcolorbox}
    \vfill
    {\large \startDate\par}
    \vfill
  \end{titlepage}

  %-----------------------------------------------------------------
  %Comienzo del TOC
  \clearpage
  \tableofcontents
  \clearpage
  %-----------------------------------------------------------------

  \section{Ejemplo de Section}

  \subsection{Ejemplo de Subsection}

  \subsubsection{Ejemplo de Subsubsection}

  \section{Ejemplo de Imagen}

  Imagen que este con el tamanio del Texto en 0.5 

  \begin{figure}[h]
    \centering
    \includegraphics[width=0.5\textwidth]{\logoPortada}
    \caption{Caption para la imagen}
  \end{figure}

  \section{Ejemplo de listas} 

  \begin{enumerate}
    \item Primer Elemento enumerado
    \item Segundo Elemento enumerado
  \end{enumerate}


  \begin{itemize}
    \item Primer Elemento sin enumerar
    \item Segundo Elemento sin enumerar
  \end{itemize}


  \section{Ejemplo de smart diagram}


  \begin{figure}[h]
    \begin{center}
      \smartdiagram[priority descriptive diagram]{
        Reconocimiento sobre el sistema,
        Dereccion de vulnerabilidades,
        Explotacion de vulnerabilidades,
        Securizacion del sistema
      }
      \caption{Flujo de trabajo} 
    \end{center}
  \end{figure}

  \section{Ejemplo de href}

  Este es un ejemplo de href \href{https://tryhackme.com}{\textbf{\color{green}HackTheBox}}


  \section{Referencia a una imagen}
 
  \begin{figure}[h]
  \centering
    \includegraphics[width=0.3\paperwidth]{\logoPortada}
    \caption{Enumeracion de servicios y versiones}
    \label{fig:servicesResults}

  \end{figure}

  \vspace{0.2cm}

  Tal y como se aprecia en la figura \ref{fig:servicesResults} de la pagina
  \pageref{fig:servicesResults}, hay un ejemplo de para hacer referencia a una
  imagen
 

  \clearpage

\end{document}
